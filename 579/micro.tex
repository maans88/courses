\documentclass{IEEEtran}

\title{Homework 4: Microservices}
\author{Maan Joseph}

\begin{document}
	\maketitle
	\begin{abstract}
		This paper covers the different styles of the microservices and their differences and similarities. Also the method of how microserices supports each one of a-loose coupling, b-agile computing and c-handling. Lastly, we will touch on the different testing strategies for microserices.   
	\end{abstract}

	\section{Introduction}
		In an event of wanting to design a very large complex program, it is better to build it piece by piece rather than having one big program. Reason being is that if feature of the program needs to be upgraded or modified, it would not require a redisgn to the whole system architecture. That is where microservices comes in place, it consist of multiple services that work independently to build up a larger software. These software services are loosely coupled and can work independent of one another \cite{ibmred}. More percisely, Microservices offer a solution to one specific task only, and it will do that task well. Microserices can be written in any language of choice and can be different from other services in the same program. The reason behind that is that microservices use the form of language-neutral application interface (API).


		\subsection{Microservices architecture style}
			In this section we discuss Microservice in more details. Microservices are small and focused, meaning that it is a software that was designed to solve only task only. Microservice should be treated as a single piece of software and should have its own source code management repository as well its own build pipeline for delivery and deployment \cite{ibmred}..
			\newline
			
			
			A microservice must be loosely coupled from one another. In an event a developer would need to update a small section of software, then he/she must be able to do so without breaking the software down and causing down-time. Therefore, microservices must be completely independent from one another and should not rely on any other microservices to run at all times \cite{ibmred}..
			\newline
			

			Language neutral is an important aspect of microservices. As mentioned before, each microservice is completely independent of one another, and if one microservice needs to be implemented in one language, then other microservice should not follow the same language. Microservices use Hyper Text Transfer Protocol (HTTP) to communicate with each other and an example of that is REST \cite{ibmred}.

		\subsection{Service-Oriented Architecture Style}
			SOA consists of a set of services. Services that are offered by a business to their customers, partners and other organizations. Therefore it is a more complex entity than Microservices. Most ofteny, SOA requires a service requester which is handled by the service provider at the business offering the SOA and possibly mediation in some cases. SOA encapsulates a set of architecural principles that are responsible for loose-coupling, reuse, compatibility and encapsulation. SOA has the technologies, tools and standards to support web services and REST services \cite{ibmred}.


	\section{Differences between Microservices architectural style and Service-Oriented architecture}
		% 1a-Define each of the Microservices architectural style and the Service-Oriented architecture. b-clearly describe the differences between them.[3pts]

		% 2. How do Microservices architecture support a-loose coupling, b-agile computing, and c-handling of complex applications? [3 pts]

		% 3. Describe 4 different testing strategies for Microservices.[4pts]


	\newpage
	\bibliography{micro}
	\bibliographystyle{plain}

\end{document}