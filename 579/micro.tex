\documentclass{IEEEtran}

\title{Homework 4: Microservices}
\author{Maan Joseph}

\begin{document}
	\maketitle
	\begin{abstract}
		This paper covers the different styles of the microservices and their differences and similarities. Also the method of how microserices supports each one of a-loose coupling, b-agile computing and c-handling. Lastly, we will touch on the different testing strategies for microserices.   
	\end{abstract}

	\section{Introduction}
		In an event of wanting to design a very large complex program, it is better to build it piece by piece rather than having one big program. Reason being is that if feature of the program needs to be upgraded or modified, it would not require a redisgn to the whole system architecture. That is where microservices comes in place, it consist of multiple services that work independently to build up a larger software. These software services are loosely coupled and can work independent of one another \cite{ibmred}. More percisely, Microservices offer a solution to one specific task only, and it will do that task well. Microserices can be written in any language of choice and can be different from other services in the same program. The reason behind that is that microservices use the form of language-neutral application interface (API).


		\subsection{Microservices architecture style}
		\subsection{Service-Oriented Architecture Style}

	\section{Differences between Microservices architectural style and Service-Oriented architecture}
		% 1a-Define each of the Microservices architectural style and the Service-Oriented architecture. b-clearly describe the differences between them.[3pts]

		% 2. How do Microservices architecture support a-loose coupling, b-agile computing, and c-handling of complex applications? [3 pts]

		% 3. Describe 4 different testing strategies for Microservices.[4pts]


	\newpage
	\bibliography{micro}
	\bibliographystyle{plain}

\end{document}