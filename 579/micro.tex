\documentclass{IEEEtran}

\title{Homework 4: Microservices}
\author{Maan Joseph}

\begin{document}
	\maketitle
	\begin{abstract}
		This paper covers the different styles of the microservices and their differences and similarities. Also the method of how microservices supports each one of a-loose coupling, b-agile computing and c-handling. Lastly, we will touch on the different testing strategies for microservices.   
	\end{abstract}

	\section{Introduction}
		In an event of wanting to design a very large complex program, it is better to build it piece by piece rather than having one big program. Reason being is that if feature of the program needs to be upgraded or modified, it would not require a redesign to the whole system architecture. That is where microservices comes in place, it consist of multiple services that work independently to build up a larger software. These software services are loosely coupled and can work independent of one another \cite{ibmred}. More precisely, Microservices offer a solution to one specific task only, and it will do that task well. Microservices can be written in any language of choice and can be different from other services in the same program. The reason behind that is that microservices use the form of language-neutral application interface (API).


		\subsection{Microservices architecture style}
			In this section we discuss Microservice in more details. Microservices are small and focused, meaning that it is a software that was designed to solve only task only. Microservice should be treated as a single piece of software and should have its own source code management repository as well its own build pipeline for delivery and deployment \cite{ibmred}..
			\newline
			
			
			A microservice must be loosely coupled from one another. In an event a developer would need to update a small section of software, then he/she must be able to do so without breaking the software down and causing down-time. Therefore, microservices must be completely independent from one another and should not rely on any other microservices to run at all times \cite{ibmred}.
			\newline
			

			Language neutral is an important aspect of microservices. As mentioned before, each microservice is completely independent of one another, and if one microservice needs to be implemented in one language, then other microservice should not follow the same language. Microservices use HyperText Transfer Protocol (HTTP) to communicate with each other and an example of that is REST \cite{ibmred}.

		\subsection{Service-Oriented Architecture Style}
			SOA consists of a set of services. Services that are offered by a business to their customers, partners and other organizations. Therefore it is a more complex entity than Microservices. Most oftenly, SOA requires a service requester which is handled by the service provider at the business offering the SOA and possibly mediation in some cases. SOA encapsulates a set of architectural principles that are responsible for loose-coupling, reuse, compatibility and encapsulation. SOA has the technologies, tools and standards to support web services and REST services \cite{ibmred}.


	\section{Differences between Microservices architectural style and Service-Oriented architecture}
		The main difference between the two is the target of users. On one hand the SOA wants to targer anyone looking for the service, therefore what it offers is more open and more likely is able to hand different kind of tasks. On the other hand, Microservices targets a specific person or a task and is much more focused on a specific task. Therefore it is safe say that Microservices were designed to focus to solve a single problem and is also focused on evolving into a bigger distributed system of microservices incrementally. The microservices applications are managed and run by one master application, which can choose to reuse the same microservice if wanted but not at a service level. Therefore it becomes more clear that SOA was designed to target complex enterprise architecture problems \cite{ibmred}. 
		\newline
		
		In the financial aspect to the company, it would be cheaper initially to implement microservices rather than SOA. SOA requires a large financial support from the beginning because it is a complex service that solves targets different tasks. It would be more financially friendly to start off with a microservice because it starts by targeting a specific task and as it evolves, the more money it is required. Therefore, another way to look at using microservices to revolutionize a large project is much like a pay-as-you go service \cite{ibmred}. 


	\section{How Microservice supports:}
		In this section we will look at the process of Microservices architecture supporting loose coupling, agile computing, and handling of complex applications.  

		\subsection{Loose Coupling}
			First we have to look at the definition of loosely-coupled. In an architectural complex system that consist of other sub-system, each of these sub-system are independent of one another, i.e. one subsystem is not dependant to work by other sub-system \cite{mell2009effectively}. 
			\newline

			That feature is an essential aspect of microservices, because they work independently to solve a specific task with one goal. Since microservices consist of a single program that targets a specific task, and this program is not dependant on a different program in the architecture, nor it would cause a downtime when it is being fixed and updated then it is considered loosely-coupled. Furthermore, microservices use Neutral language API for communication purposes, this also improves the loosely-coupled featured because it is not tied to other languages used in the platform \cite{ibmred}. 
		\subsection{Agile Computing}
			Agile computing is a set of principles in software development. The set of principles guide the development of the project by certain guidelines and help the project evolves through collaborative effort and exchange of information and opinion between developers \cite{mell2009effectively}. 
			\newline

			Microservices support agile computing through track and plan services by the IBM Bluemix DevOps. 
			DevOps also provide a web integrated IDE, source control management and lastly automated builds and deployments through pipeline along with agile computing \cite{ibmred}. 

		\subsection{Handling of Complex Application}
			Microservices handle complex application by following differents phases or stages of development. First, there is a set of requirements that are set and must be followed in order to build a complex application, these requirements would be fundamental building blocks in order for the application to evolve. Secondly, Microservices offer different types of testing of the application throughout the development phases. It also provides deployment of the software to the customer when the time is appropriate. It also provides operational monitoring of the service throughout the development and deployment process. 

	\section{Testing Strategies}
		Microservices offer different types of testing services, which consist of Unit testing, Integration testing, Component testing, Contract testing, End-to-end testing and lastly performance testing. We will examine four of these testings below: 

		\subsection{Unit Testing}
			Unit testing is offered by microservices the same way it is offered by other services. Unit testing targets and examines the smallest piece of testable software piece in the application, however there is no set size of "smallest" piece of testable software. Unit testing aims to test whether the piece of software being tested to determine if it actually behaves as it is supposed to the guidelines set. Furthermore, unit testing is usually conducted by the developer team that are working on the application \cite{ibmred}. 
			\newline

		\subsection{Integration Testing}
			As the application development evolves, integration testing emerges to test the newly formed modules of the application to determine that the communication between the modules work as they are supposed to, without any errors or confusion or assumptions. In microservices, integration testing comes in place to test the application component integration with other external components they might use, for data storage as an example \cite{ibmred}. 
			\newline

		\subsection{End-End Testing}
			This method of testing is conducted when the application is at its final production phases before it is to be delivered to the customer. It tests the application as a whole against the set requirements that are set by the customer and makes the determination whether the application developed meets the standards and requirements set by the user. In microservices, this method of testing offers to cover the gaps between the services that exist within the application by making sure the form of communication between them is not broken. Furthermore, it ensures the firewall and other network settings are set properly \cite{ibmred}.
			\newline

		\subsection{Performance Testing}
			Performance testing is conducted on microservices between one service to another, more specifically, it tests the calls between service-to-service and usually targets one service. The goal of the testing method to have the performance issue of the service under control. In microservices, performance testing is conducted by having as much real data as possible and perform the performance test regularly \cite{ibmred}.
			\newline

	\section{Conclusion}
		We looked at Microservices and Service-Oriented-Architecture. Both of these infrastructure aim to solve tasks. They differ by the way they accomplish their goals. Microservices work with a goal to solve a specific task and does it well, while on the other hand, SOA aims to solve a more complex task, and it would be a service that is offered by a business to solve a bigger complex architecture. Using microservices is not as costly initially because it targets smaller tasks then it would cost more as it evolves into a bigger platform. SOA would cost more financially initially, because it is a much more complex service. 
		% 1a-Define each of the Microservices architectural style and the Service-Oriented architecture. b-clearly describe the differences between them.[3pts]

		% 2. How do Microservices architecture support a-loose coupling, b-agile computing, and c-handling of complex applications? [3 pts]

		% 3. Describe 4 different testing strategies for Microservices.[4pts]

	\newpage
	\bibliography{micro}
	\bibliographystyle{plain}

\end{document}