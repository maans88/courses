\documentclass{IEEEtran}

\title{Homework 4: Microservices}
\author{Maan Joseph}

\begin{document}
	\maketitle
	\begin{abstract}
		This paper covers the different styles of the microservices and their differences and similarities. Also the method of how microserices supports each one of a-loose coupling, b-agile computing and c-handling. Lastly, we will touch on the different testing strategies for microserices.   
	\end{abstract}

	\section{Introduction}
		In an event of wanting to design a very large complex program, it is better to build it piece by piece rather than having one big program. Reason being is that if feature of the program needs to be upgraded or modified, it would not require a redisgn to the whole system architecture. That is where microservices comes in place, it consist of multiple services that work independently to build up a larger software. These software services are loosely coupled and can work independent of one another \cite{ibmred}. More percisely, Microservices offer a solution to one specific task only, and it will do that task well. Microserices can be written in any language of choice and can be different from other services in the same program. The reason behind that is that microservices use the form of language-neutral application interface (API).


		\subsection{Microservices architecture style}
			In this section we discuss Microservice in more details. Microservices are small and focused, meaning that it is a software that was designed to solve only task only. Microservice should be treated as a single piece of software and should have its own source code management repository as well its own build pipeline for delivery and deployment \cite{ibmred}..
			\newline
			
			
			A microservice must be loosely coupled from one another. In an event a developer would need to update a small section of software, then he/she must be able to do so without breaking the software down and causing down-time. Therefore, microservices must be completely independent from one another and should not rely on any other microservices to run at all times \cite{ibmred}.
			\newline
			

			Language neutral is an important aspect of microservices. As mentioned before, each microservice is completely independent of one another, and if one microservice needs to be implemented in one language, then other microservice should not follow the same language. Microservices use Hyper Text Transfer Protocol (HTTP) to communicate with each other and an example of that is REST \cite{ibmred}.

		\subsection{Service-Oriented Architecture Style}
			SOA consists of a set of services. Services that are offered by a business to their customers, partners and other organizations. Therefore it is a more complex entity than Microservices. Most ofteny, SOA requires a service requester which is handled by the service provider at the business offering the SOA and possibly mediation in some cases. SOA encapsulates a set of architecural principles that are responsible for loose-coupling, reuse, compatibility and encapsulation. SOA has the technologies, tools and standards to support web services and REST services \cite{ibmred}.


	\section{Differences between Microservices architectural style and Service-Oriented architecture}
		The main difference between the two is the target of users. On one hand the SOA wants to targer anyone looking for the service, therefore what it offers is more open and more likely is able to hand different kind of tasks. On the other hand, Microservices targets a specific person or a task and is much more focused on a specific task. Therefore it is safe say that Microservices were designed to focus to solve a single problem and is also focused on evolving into a bigger distributed system of microservices incrementally. The microservices applications are managed and run by one master application, which can choose to reuse the same microservice if wanted but not at a service level. Therefore it becomes more clear that SOA was designed to targest complex enterprise architecture problems \cite{ibmred}. 
		\newline
		
		In the financial aspect to the company, it would be cheaper initially to implement microservices rather than SOA. SOA requires a large financial support from the beginning because it is a complex service that solves targets different tasks. It would be more financially friendly to start off with a microservice because it starts by targeting a specific task and as it evolves, the more money it is required. Therefore, another way to look at using microservices to evolutionizse a large project is much like a pay-as-you go service \cite{ibmred}. 


	\section{How Microservice supprots:}
		In this section we will look at the process of Microservices architecture supporting a-loose coupling, b-agile computing, and c-handling of complex applications.  

		\subsection{a-loose coupling}
			First we have to look at the definition of loosely-coupled. In an archritecture complex system that consist of other sub-system, each of these sub-system are independent of one another, i.e. one subsystem is not dependant to work by other sub-system. Which is an essential aspect of microservices, because they work independently to solve a specific task with one goal. Since microservices consist of a single program that targets a specific task, and this program is not dependant on a different program in the architcure, nor it would cause a downtime when it is being fixed and updated then it is considered loosely-coupled. Further more, microservices use Neutral language API for communication purposes, this also improves the loosely-coupled featured because it is not tied to other languages used in the platform. 
		\subsection{b-agile computing}
			
		\subsection{c-handling of Complex Application}
	

	\section{Testing Strategies}
	\section{Conclusion}
		We looked at Microservices and Serice-Oriented-Architcure. Both of these infrastructure aim to solve tasks. They differ by the way they accomplish their gaols. Microserices work with a goal to solve a specific task and does it well, while on the other hand, SOA aims to solve a more complex task, and it would be a service that is offered by a business to solve a bigger complex architecture. Using microservices is not as costly initially because it targets smaller tasks then it would cost more as it evolves into a bigger platform. SOA would cost more financially initially, because it is a much more complex service. 
		% 1a-Define each of the Microservices architectural style and the Service-Oriented architecture. b-clearly describe the differences between them.[3pts]

		% 2. How do Microservices architecture support a-loose coupling, b-agile computing, and c-handling of complex applications? [3 pts]

		% 3. Describe 4 different testing strategies for Microservices.[4pts]


	\newpage
	\bibliography{micro}
	\bibliographystyle{plain}

\end{document}